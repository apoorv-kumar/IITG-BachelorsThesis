
\lhead[\chaptername~\thechapter]{\rightmark}


\rhead[\leftmark]{}


\lfoot[\thepage]{}


\cfoot{}


\rfoot[]{\thepage}


\chapter{Why NoCs / Cost Considerations.}


\section{Overview}

Before we can start on the best/optimal architecture we have to specify
on what grounds we would be measuring performance. Further we have
to ensure that the NOC architecture does improve upon alternative
architectures on these grounds.

Under most widely accepted norms , we analyze the generic cost in
terms of area and power. 

NoCs help us overcome the common issues faced by currently used architectures.
This chapter discusses the problems in brief.


\section{Power}

Power has become the most critical constraint in the design of many
systems, from high-performance servers to em- bedded battery-operated
devices. Recognizing the need to target increasing power consumption
and design complexity in these systems, designers have turned to multi-core
architectures such as chip multiprocessors (CMPs) and multiprocessor
systems-on-a-chip (MPSoCs).


\subsection{Network Power}

Both the processor cores as well as the NoC communication fabric can
be modeled as a fully connected directed graph \textit{\small G =
(N, L)} where N is the set of nodes and L is the set of links in G.
This holds true for the ordinary multi-processors as well , and uni-cores
just behave as a trivial subset. The only difference lies in how the
fabric behaves. Whereas in unicores this power can be safely neglected
, this power is included in the shared-bus power consumption for the
ordinary multicores.

There are many frameworks for the estimation of this power. Some of
them being LUNA and Orion.

Most commonly used ``proxy-parameters'' for the power consumption
include link-utilization. Dynamic network power, is a function of
activity/utilization and energy costs (constants) for each of the
key router components. Using link utilization as an abstraction for
network power , the level of activity at a network link is used as
a measure the overall power consumption of that network router and
link.\cite{Eisley:2004:HPA:1023833.1023849}


\subsection{Processor Power }

Since we can model the processor cores using the same above graph,
the same frameworks would work fine here as well. But in contrast
to the network model , the data transfer over nodes is much cheaper
and faster process. Resource utilization is usually used as a proxy
for power, similar to the way we abstract network power through link
utilization in Section 1.2.1, we abstract the power consumption of
a processor by the utilizations of individual resources. The summation
of the energy costs of each component (functional units, register
file, caches etc.) is captured by each respective utilization function. 

In the case of the network fabric, the utilization of all of the components
is approximately equal because message flowing in networks consume
roughly the same amount of energy per hop. Estimates of individual
network components\textquoteright{} energy consumptions are thus not
necessary because it is a relative power measure; constant factors
can be eliminated. However, in the case of a processor pipeline, each
instruction will not consume the same amount of energy. The component
utilizations hence cannot be removed and abstracted as a single utilization.
Relative estimates are thus required in order to obtain a relatively
accurate estimation of processor power.


\section{Area}

With the failure of Dennard scaling and thus slowed supply voltage
, scaling core count scaling may be in jeopardy, which would leave
the community with no clear scaling path to exploit continued transistor
count increases. In any case , preserving the transistors for the
purpose of actual computing is a priority for any designer.For instance,
by increasing the buffer size at each input channel from 2 to 3 words,
the router area of a 4x4 NoC increases by 30\% or more\cite{Ogras:2005:KRP:1084834.1084856}.To
this end we like to keep the buffers (that add absolutely nothing
to computations) to the minimum.Buffer consumes much of silicon and
power and this is likely to reduce performance. 
