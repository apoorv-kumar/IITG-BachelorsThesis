
\chapter*{Introduction}

\addcontentsline{toc}{chapter}{Abstract} 

Intel projects the availability of 100 billion transistors on a 300mm
die by 2015. This allows to integrate thousands of processors or equivalent
logic gates on a single die. Unfortunately , we have hit various roadblocks
on our way to performance enhancement. Heat wall , synchronization
problem , crosstalk at sub-20nm levels , all equally fatal. This forces
us to look towards horizontal rather than vertical scaling. We use
many transistors to increase the working units, rather than improving
one unit. The major challenge lies in trying to map our applications
initially designed for serial execution on these massively parallel
machines. Further since our computers need to handle dynamic loads
, dynamic mapping comes out as the natural preference for mapping.

On-chip packet-switched network have been proposed as a solution for
the problem of global interconnect in deep sub-micron VLSI Systems
on Chip (SoC). Networks on Chip (NoC) can address and contain major
physical issues such as synchronization, noise, error correction and
speed optimization. NoC can also improve design productivity by supporting
modularity and reuse of complex cores, thus enabling a higher level
of abstraction in architectural modeling of future systems .

In this report we discuss the various paradigms available for running
real life applications on the NoC architectures. First we highlight
the compulsions that forced us to switch from the currently prevalent
architectures. Then we justify why we set aside other alternative
architectures in favour of the network based onces. Once we are on
firm footing as to why we have chosen the aforementioned paradigm
, we discuss the parameters that define scalability ,  low cost and
good performance. Further we discuss in details how we plan to run
our applications on the many-cores , that don't nearly resemble the
bus-based architecture we are so used to program for. We show how
centralized mapping has very binding limitations , and that decetralized
mapping is the best way for the distant future. 
